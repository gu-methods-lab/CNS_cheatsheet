\documentclass[a4paper, 10pt]{article}
\usepackage{tabulary}
\usepackage{floatrow}
\usepackage{setspace}
\usepackage[misc]{ifsym}
\usepackage[utf8]{}
\usepackage{amsmath,amssymb}
\usepackage[]{hyperref} 
\usepackage[numbers]{natbib}
\bibliographystyle{plainnat}
\usepackage{caption}

\usepackage{titlesec}
\titleformat*{\section}{\large\bfseries}
\renewcommand*{\bibfont}{\footnotesize}

\begin{document}


{\centering\doublespacing
\Large{\textbf{A cheatsheet for computational neuroscientists}} 

\normalsize{\textbf{Yifan Gu} \\
\small{{School of Physics and ARC Centre of Excellence for Integrative Brain Function, \\
University of Sydney, NSW 2006, Australia} }	
\date{\today} 
\par
}
}


\begin{table}[h]
\captionsetup{font=small}
\footnotesize
\centering
\begin{tabular}[c]{p{3cm}p{6cm}}
\hline \\
770-2900 $\mu$m/ms &  conduction speed of action potential in myelinated axons in the cortex  \cite{debanne2011axon} \\
250-380 $\mu$m/ms & conduction speed of action potential in unmyelinated axons in the cortex  \cite{debanne2011axon} \\
140 $\mu$m/ms &  propagating speed of epileptiform waves in disinhibited hippocampal slices  \cite{miles1988spread} \\
60-90 $\mu$m/ms &  propagating speed of epileptiform waves in disinhibited neocortical slices  \cite{pinto2005initiation, chervin1988periodicity} \\
6-10 $\mu$m/ms & propagating speed of population activation in neocortical slices under conditions of
unaltered excitability   \cite{wu1999propagating} \\
\hline
\end{tabular}
\caption{Activity propagation speed}
\end{table}


\begin{table}[h]
\captionsetup{font=small}
\footnotesize
\centering
\begin{tabular}[c]{p{3cm}p{6cm}}
\hline \\
200-450 $\times 10^{-6}$  $\mu$m$^{-3}$ & density of pyramidal neurons in rodent hippocampus \cite{jinno2010stereological} \\
50-60  $\mu$m & thickness of stratum pyramidal of rodent hippocampus (where pyramidal neurons lie) \cite{ghafari2014prenatal} \\
2261 mm$^2$ & total cortical surface area of a hemisphere of a galago \citep{collins2010neuron} \\
127 $\times 10^{6}$ & estimated number of neurons in the above cortical area \citep{collins2010neuron} \\
18577 mm$^2$ & total cortical surface area of a hemisphere of a Baboon \citep{collins2010neuron} \\
2.36 $\times 10^{9}$ & estimated number of neurons in the above cortical area \citep{collins2010neuron} \\
\hline
\end{tabular}
\caption{Neuron density}
\end{table}

\begin{table}[h]
\captionsetup{font=small}
\footnotesize
\centering
\begin{tabular}[c]{p{3cm}p{6cm}}
\hline \\
0.77 mV   & average peak unitary EPSP amplitude from baseline recorded from layer 5 pyramidal neurons in the rat visual cortex; note that the distribution is lognormal  with some large EPSPs ($> 5$ mV)  \cite{song2005highly} \\
$21.1 \pm 10.6$ nS & average membrane conductance change between Up and Down state during spontaneous activity in the prefrontal cortex in anesthetized ferrets \cite{haider2006neocortical} \\
\hline
\end{tabular}
\caption{Synaptic activity}
\end{table}


\bibliography{CNS_cheatsheet}

\end{document}






